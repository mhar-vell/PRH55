\chapter{Caracterização do problema}
Atualmente veículos de inspeção submarina necessitam de um barco de suporte para lançamento e recuperação, o que limita a operação para apenas condições de mar favoráveis. No projeto será desenvolvido um veículo, o que permite o veículo ser lançado de plataformas e outros navios. Esse desenvolvimento reduz a complexidade da operação e possibilita a inspeção independentemente das condições do mar. 
O sistema de navegação será um sistema híbrido, ou seja, a arquitetura de controle será projetado de tal forma que se possa mudar de semi-automático para o modo de ROV. O que permitirá um operador controlar o veículo enquanto assistido por todas as funcionalidades da navegação autônoma. O projeto usará também técnicas de navegação adaptativa, sendo assim capaz de reagir às informações dos sensors e adaptar-se.
No projeto está previsto o desenvolvimento de uma interface intuitiva, permitindo a fácil configuração do veículo e operação, assim, dispensando a necessidade de pessoal especializado para as operações. 
