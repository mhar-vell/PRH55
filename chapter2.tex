\chapter{Caracterização do problema}
Atualmente veículos de inspeção submarina são controlados por pessoas altamente qualificadas independentemente das atividades realizadas aumentando os custos de operação e o tempo para a mobilização deste pessoal pois é necessário um navio de apoio para levar o ROV e a equipe envolvida neste processo, além de depender de condições climáticas o que desfavorece as instalações com tempo de uso acima de 25 anos que dependem de uma manutenção preventiva mais constante. Mediante a estas características de estrutura e logística propõe-se uma aplicação local da tecnologia de inspeção de estrutura submersas, visando a utilização de um ROV fixo em uma instalação para desenvolver inspeções mais contínuas e específicas com o intuito de aumentar a confiabilidade das instalações e a vida útil das estruturas submersas, podendo ser controlado por pessoas qualificadas e autorizadas que pertencem a este local de trabalho e realizando inspeções semi-autônomas através de uma configuração realizada previamente por pessoal qualificado, reduzindo a complexidade da manipulação do ROV com uma interface mais intuitiva, consequentemente o nível de instrução dos operadores e o custo hora-homem.   
