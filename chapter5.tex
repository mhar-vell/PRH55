\chapter{Resultados esperados}
Espera-se que ao final deste projeto obtenha-se um ROV que execute algumas ações autonomamente através de um programa supervisório de baixo custo instalado em um computador, com uma interface intuitiva, um controle preciso e de fácil adaptabilidade, proporcionando o controle pleno do OpenROV, de comum acordo com as simulações no GAZEBO, e o desenvolvimento de um TCC. 

