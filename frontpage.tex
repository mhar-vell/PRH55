%%%%%%%%%%%%%%%%%%%%%%%%%%%%%%%%%%%%%%%%%%%%%%%%%%%%%%%%%%%%%%%%%%%%%%%%%%
%   This is frontpage.tex file needed for the mhar-vell.cls file.  You   %
%  have to  put this file in the same directory with your thesis files.  %
%                Written by M. Imran 2001/06/18                          % 
%                 No Copyright for this file                             % 
%                 Save your time and enjoy it                            % 
%                                                                        % 
%%%%%%%%%%%%%%%%%%%%%%%%%%%%%%%%%%%%%%%%%%%%%%%%%%%%%%%%%%%%%%%%%%%%%%%%%%%
%%%%%%%%%%%%%%%%%%%%%%%%%%%%%%%%%%%%%%%%%%%%%%%%%%%%%%%%%%%%%%%%%%%%%%%%%%%
%%%%%%%%%%%%%%%%           The title page           %%%%%%%%%%%%%%%%%%%%%%%  
%%%%%%%%%%%%%%%%%%%%%%%%%%%%%%%%%%%%%%%%%%%%%%%%%%%%%%%%%%%%%%%%%%%%%%%%%%%
\pagenumbering{roman}
%\pagenumbering{arabic}

\setcounter{page}{1}

\newpage

\thispagestyle{empty}

\begin{center}
  \vspace*{0.8cm}
  {\Large \bf Proposta de projeto de pesquisa do programa PRH 55 da ANP com ênfase no setor de petróleo, gás e biocombustíveis, como parte do processo de formação científica.}

  \vspace*{2.5cm}
 % {\LARGE\bf Marco Antônio dos Reis}
  %\vspace*{3cm}
  %\vfill

  {\Large Robô de inspeção com capacidade de realizar operações semi-autônomas em estruturas subaquáticas utilizando ativos nacionais e de baixo custo.}
  %\\
           \vspace*{2.5cm}
  
  {\bf ---------------------------------------\\
  Luiz Paulo de Souza Santos}
  
  \vspace*{1.5cm}
  {\bf --------------------------------------------------\\
  Marco Reis, M.Sc. (Orientador)}
  
   \vspace*{1.5cm}
  {\bf ------------------------------------------------------------------\\
  Geovani Mimoso , M.Sc. (Co-orientador)}
  
  %{\LARGE\bf Marco Antônio dos Reis}
  % Put your university logo here if you wish.
   \begin{center}
   %\includegraphics{DU_2-col_sml.eps}
   \end{center}
	\vspace*{- 0.3cm}

  {Programa de Formação de Recursos Humanos da ANP – PRH 55\\  
          [-0.3cm] Linha de pesquisa em sistemas computacionais\\
          [-3mm] Senai Cimatec\\
          [-3mm] Salvador - Bahia\\
          [-3mm] março de 2016}
\end{center}

%%%%%%%%%%%%%%%%%%%%%%%%%%%%%%%%%%%%%%%%%%%%%%%%%%%%%%%%%%%%%%%%%%%%%%%%%%%
%%%%%%%%%%%%%%%% The dedication page, of you have one  %%%%%%%%%%%%%%%%%%%%  
%%%%%%%%%%%%%%%%%%%%%%%%%%%%%%%%%%%%%%%%%%%%%%%%%%%%%%%%%%%%%%%%%%%%%%%%%%%
\newpage
\thispagestyle{empty}
\begin{center}
 \vspace*{2cm}
 % \textit{\LARGE {Dedicated to}}\\ 
 %Someone here
\end{center}


%%%%%%%%%%%%%%%%%%%%%%%%%%%%%%%%%%%%%%%%%%%%%%%%%%%%%%%%%%%%%%%%%%%%%%%%%%%
%%%%%%%%%%%%%%%%%%           The abstract page         %%%%%%%%%%%%%%%%%%%%  
%%%%%%%%%%%%%%%%%%%%%%%%%%%%%%%%%%%%%%%%%%%%%%%%%%%%%%%%%%%%%%%%%%%%%%%%%%%
\newpage
\thispagestyle{empty}
\addcontentsline{toc}{chapter}{\numberline{}Resumo}
\begin{center}
  \textbf{\Large Robô de inspeção com capacidade de realizar operações semi-autônomas em estruturas subaquáticas utilizando ativos nacionais e de baixo custo.
  }

  \vspace*{1cm}
  \textbf{\large Luiz Paulo de Souza Santos}

  \vspace*{2cm}
 % {\large Submetido para proposta de projeto ao processo de admissão no Doutorado de Mecatrônica\\ janeiro de 2016}

  %\vspace*{2cm}
  \textbf{\large Resumo}
\end{center}
%Durante vários anos, a inspeção de estruturas submarinas tem sido feito regularmente por ROVs ou mergulhadores em alta profundidade. Além de ser uma forma onerosa de inspeção, em alguns casos, devido às condições meteorológicas e outros fatores que dificultam a inspeção. Como alternativa para o uso de mergulhadores, a inspeção por veículos remotamente controlados é apresentada como uma inspeção limitada, devido ao tempo despendido para a realização de tarefas simples e repetidas, e que muitas vezes não são reprodutivas e não alcançam um bom desempenho quanto a repetibilidade.
%Desta forma uma inspeção por ROVs com operações semi-autônomas torna-se uma solução vantajosa quanto à redução dos custos, riscos relacionados aos recursos e aumento de produtividade. 



Desde o surgimento da extração de petróleo e gás em ambientes subaquáticos há a necessidade de inspecionar as estruturas submersas devido a grande salinidade do mar, a movimentação da plataforma causadas pelas ondas e ao tempo de utilização das plataformas. Estas inspeções são realizadas atualmente por mergulhadores e ROV's  (Remotely Operated Vehicle) no intuito de evitar a corrosão, trincas ou rompimento das tubulações e/ou cabos que ligam a plataforma ao inferior do oceano. As altas profundidades proporcionam uma variação muito grande de pressão e temperatura as quais um ser humano não deve ser exposto, diante disso os ROV's tem uma grande vantagem, entretanto algumas limitações como os custos operacionais, e condições meteorológicas são fatores cruciais para o desenvolvimento dessas vistorias pois os processos atuais dependem de um navio de apoio offshore e mão de obra altamente especializadas.
A rotina de manutenção nas atividades em plataformas é repetitiva pois essas estruturas são fixas e quanto mais velha é a instalação mais rotinas devem ser realizadas o que faz uma aplicação local e semi autônoma de um Veículo operado remotamente um ótimo método de redução de custos e um aumento na confiabilidade das instalações.

O projeto será realizado mediante um ambiente de simulação utilizando o framework ROS e o software GAZEBO. Com isso espera-se apresentar uma alternativa à metodologia de inspeção em estruturas submarinas utilizada hoje em dia. A continuidade da pesquisa no uso de uma solução mecânica já desenvolvida é algo vital para este projeto, logo será utilizado o OpenRov, um sistema mecânico já utilizado em pesquisas anteriores para inspeções em estruturas submarinas.
Este projeto visa desenvolver um ROV, capaz de realizar inspeção em integridades estruturais, como plataformas e tubulações realizando operações específicas semi-automaticamente e será desenvolvido em duas fases consecutivas de 1 anos cada. Na primeira fase será desenvolvida um protótipo funcional e de baixo custo visando a Montagem mecânica, eletrônica e o desenvolvimento do Software de controle da movimentação manual e na fase consecutiva será desenvolvido o algoritmo de análise visual para a inspeção em tubulação e a movimentação semiautônoma para a realização de inspeção submarina.

 

\vspace*{1cm}
\noindent \textbf{Palavras chaves: }robô de inspeção, inspeção subaquática.


%%%%%%%%%%%%%%%%%%%%%%%%%%%%%%%%%%%%%%%%%%%%%%%%%%%%%%%%%%%%%%%%%%%%%%%%%%%
%%%%%%%%%%%%%%%%%%          The declaration page         %%%%%%%%%%%%%%%%%%  
%%%%%%%%%%%%%%%%%%%%%%%%%%%%%%%%%%%%%%%%%%%%%%%%%%%%%%%%%%%%%%%%%%%%%%%%%%%
\chapter*{Declaração}
\addcontentsline{toc}{chapter}{\numberline{}Declaração}
O projeto nesta proposta é baseado na pesquisa realizada pelo Brazilian Institute of Robotics - BIR / Senai Cimatec. A pesquisa versa na área da inspeção de estruturas subaquáticas. Referências contidas nela são estritamente de responsabilidade do autor.\\
\vspace*{8cm}
\vfill
\noindent \textbf{Copyright\copyright\; 2016 by Luiz Paulo de Souza Santos}.\\
``Os direitos de autoria deste projeto recai sobre o autor. Citações não devem ser publicadas sem o consentimento prévio por escrito do autor e informações oriundas da mesma devem ser reconhecidas e referenciadas''.



%%%%%%%%%%%%%%%%%%%%%%%%%%%%%%%%%%%%%%%%%%%%%%%%%%%%%%%%%%%%%%%%%%%%%%%%%%%
%%%%%%%%%%%%%%%%%%     The acknowledgements page         %%%%%%%%%%%%%%%%%%  
%%%%%%%%%%%%%%%%%%%%%%%%%%%%%%%%%%%%%%%%%%%%%%%%%%%%%%%%%%%%%%%%%%%%%%%%%%%
%\chapter*{Agradecimentos}
%\addcontentsline{toc}{chapter}{\numberline{}Acknowledgements}
%Thank to someone who prepared this template. Thank to someone who
%prepared this template. Thank to someone who prepared this template.
%Thank to someone who prepared this template. Thank to someone who
%prepared this template. Thank to someone who prepared this template.
%Thank to someone who prepared this template. Thank to someone who
%prepared this template. Thank to someone who prepared this
%template.Thank to someone who prepared this template. Thank to someone
%who prepared this template. Thank to someone who prepared this
%template. Thank to someone who prepared this template. Thank to
%someone who prepared this template. Thank to someone who prepared this
%template. Thank to someone who prepared this template. Thank to
%someone who prepared this template.

%%%%%%%%%%%%%%%%%%%%%%%%%%%%%%%%%%%%%%%%%%%%%%%%%%%%%%%%%%%%%%%%%%%%%%%%%%%
%%%%%%%%    tableofcontents, listoffigures and listoftables       %%%%%%%%%
%%%%%%%%        Command if you do not have  them                  %%%%%%%%%
%%%%%%%%%%%%%%%%%%%%%%%%%%%%%%%%%%%%%%%%%%%%%%%%%%%%%%%%%%%%%%%%%%%%%%%%%%%
\tableofcontents
\listoffigures
\listoftables
\clearpage


%%%%%%%%%%%%%%%%%%%%%%%%%%%%%%%%%%%%%%%%%%%%%%%%%%%%%%%%%%%%%%%%%%%%%%%%%%%
%%%%%%%%%%%%%%%%%%%%%%   END OF FRONT PAGE %%%%%%%%%%%%%%%%%%%%%%%%%%%%%%%%
%%%%%%%%%%%%%%%%%%%%%%%%%%%%%%%%%%%%%%%%%%%%%%%%%%%%%%%%%%%%%%%%%%%%%%%%%%%









