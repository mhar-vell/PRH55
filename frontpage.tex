%%%%%%%%%%%%%%%%%%%%%%%%%%%%%%%%%%%%%%%%%%%%%%%%%%%%%%%%%%%%%%%%%%%%%%%%%%
%   This is frontpage.tex file needed for the dmathesis.cls file.  You   %
%  have to  put this file in the same directory with your thesis files.  %
%                Written by M. Imran 2001/06/18                          % 
%                 No Copyright for this file                             % 
%                 Save your time and enjoy it                            % 
%                                                                        % 
%%%%%%%%%%%%%%%%%%%%%%%%%%%%%%%%%%%%%%%%%%%%%%%%%%%%%%%%%%%%%%%%%%%%%%%%%%%
%%%%%%%%%%%%%%%%%%%%%%%%%%%%%%%%%%%%%%%%%%%%%%%%%%%%%%%%%%%%%%%%%%%%%%%%%%%
%%%%%%%%%%%%%%%%           The title page           %%%%%%%%%%%%%%%%%%%%%%%  
%%%%%%%%%%%%%%%%%%%%%%%%%%%%%%%%%%%%%%%%%%%%%%%%%%%%%%%%%%%%%%%%%%%%%%%%%%%
\pagenumbering{roman}
%\pagenumbering{arabic}

\setcounter{page}{1}

\newpage

\thispagestyle{empty}

\begin{center}
  \vspace*{0.8cm}
  {\Large \bf Proposta de projeto de conclusão do curso de pós-gradução \\
  para o Doutorado em Mecatrônica da UFBA, como parte do processo de admissão discente em 2015.2.}

  \vspace*{2.5cm}
 % {\LARGE\bf Marco Antônio dos Reis}
  %\vspace*{3cm}
  %\vfill

  {\Large Navegação autônoma para robô de inspeção de linhas de transmissão de 138kV utilizando aprendizagem de redes bayesiana e Q-learning para detecção e transposição de objetos e tento partido.}
  %\\
         %[1mm] no Doutorado em Mecatrônica da UFBA, como parte do processo de admissão discente em 2015.2}
  \vspace*{2.5cm}
  
  {\bf ---------------------------------------\\
  Marco Antonio dos Reis}
  
  \vspace*{1.5cm}
  {\bf --------------------------------------------------\\
  Dr. Leizer Schnitman (Orientador)}
  
   \vspace*{1.5cm}
  {\bf ------------------------------------------------------------------\\
  Dr. Herman Augusto Lepikson (Co-orientador)}
  
  %{\LARGE\bf Marco Antônio dos Reis}
  % Put your university logo here if you wish.
   \begin{center}
   %\includegraphics{DU_2-col_sml.eps}
   \end{center}
	\vspace*{0.3cm}

  {Programa de Pós-graduação em Mecatrônica\\
          [-3mm] Linha de pesquisa em sistemas computacionais\\
          [-3mm] Universidade Federal da Bahia\\
          [-3mm] Salvador - Bahia\\
          [-3mm] janeiro de 2016}
\end{center}

%%%%%%%%%%%%%%%%%%%%%%%%%%%%%%%%%%%%%%%%%%%%%%%%%%%%%%%%%%%%%%%%%%%%%%%%%%%
%%%%%%%%%%%%%%%% The dedication page, of you have one  %%%%%%%%%%%%%%%%%%%%  
%%%%%%%%%%%%%%%%%%%%%%%%%%%%%%%%%%%%%%%%%%%%%%%%%%%%%%%%%%%%%%%%%%%%%%%%%%%
\newpage
\thispagestyle{empty}
\begin{center}
 \vspace*{2cm}
 % \textit{\LARGE {Dedicated to}}\\ 
 %Someone here
\end{center}


%%%%%%%%%%%%%%%%%%%%%%%%%%%%%%%%%%%%%%%%%%%%%%%%%%%%%%%%%%%%%%%%%%%%%%%%%%%
%%%%%%%%%%%%%%%%%%           The abstract page         %%%%%%%%%%%%%%%%%%%%  
%%%%%%%%%%%%%%%%%%%%%%%%%%%%%%%%%%%%%%%%%%%%%%%%%%%%%%%%%%%%%%%%%%%%%%%%%%%
\newpage
\thispagestyle{empty}
\addcontentsline{toc}{chapter}{\numberline{}Resumo}
\begin{center}
  \textbf{\Large Navegação autônoma para robô de inspeção de linhas de transmissão de 138kV utilizando aprendizagem de redes bayesiana e Q-learning para detecção e transposição de objetos e tento partido.
  }

  \vspace*{1cm}
  \textbf{\large Marco Antônio dos Reis}

  \vspace*{2cm}
 % {\large Submetido para proposta de projeto ao processo de admissão no Doutorado de Mecatrônica\\ janeiro de 2016}

  %\vspace*{2cm}
  \textbf{\large Resumo}
\end{center}
Durante vários anos, a inspeção de linhas de transmissão de alta tensão tem sido feito regularmente por aeronaves tripuladas, realizando vôos a baixa altitude e perto de linhas de transmissão de alta tensão energizadas. Além de ser uma forma onerosa de inspeção, em alguns casos, devido às condições meteorológicas e outros fatores que dificultam o sobrevoo, a tripulação pode estar sujeita a riscos associados com a tarefa. Como alternativa para o uso de aeronaves, a inspeção por veículos terrestres é apresentada como uma inspeção limitada, devido ao acesso do terreno e do ângulo de visão desfavorável. Desta forma uma inspeção por sistemas robóticos torna-se uma solução vantajosa quanto à redução dos custos, riscos relacionados aos recursos e aumento de produtividade. Este projeto de tese tem como objetivo propor um sistema de navegação autônoma para detecção e transposição de objetos e tento partido realizando uma inspeção visual e térmica dos cabos das linhas de transmissão, identificando e registrando os pontos quentes presentes nos cabos e nos objetos, assim como transpondo objetos quando necessários para a sua locomoção de forma autônoma. O sistema será implementado na forma de uma simulação utilizando o framework ROS e o software GAZEBO. Com isso espera-se apresentar uma alternativa à metodologia de inspeção em linhas de transmissão energizadas utilizada hoje em dia. A continuidade da pesquisa no uso de uma solução mecânica já desenvolvida é algo vital para este projeto, logo será utilizado um sistema mecânico já utilizado em pesquisas anteriores para inspeções em linhas de transmissão.

\vspace*{1cm}
\noindent \textbf{Palavras chaves: }navegação autônoma, robô de inspeção de linhas de transmissão, aprendizagem de máquinas, redes bayesiana, Q-learning.


%%%%%%%%%%%%%%%%%%%%%%%%%%%%%%%%%%%%%%%%%%%%%%%%%%%%%%%%%%%%%%%%%%%%%%%%%%%
%%%%%%%%%%%%%%%%%%          The declaration page         %%%%%%%%%%%%%%%%%%  
%%%%%%%%%%%%%%%%%%%%%%%%%%%%%%%%%%%%%%%%%%%%%%%%%%%%%%%%%%%%%%%%%%%%%%%%%%%
\chapter*{Declaração}
\addcontentsline{toc}{chapter}{\numberline{}Declaração}
O projeto nesta proposta de tese é baseado na pesquisa realizada pelo Brazilian Institute of Robotics - BIR / Senai Cimatec. A pesquisa versa na área da inspeção de linhas de transmissão de alta tensão de 138kV. Esta proposta também faz parte como um dos subitens para uma prospecção de projeto junto a ANEEL. Referências contidas nela são estritamente de responsabilidade do autor.\\
\vspace*{8cm}
\vfill
\noindent \textbf{Copyright\copyright\; 2016 by MARCO REIS}.\\
``Os direitos de autoria deste projeto recai sobre o autor. Citações não devem ser publicadas sem o consentimento prévio por escrito do autor e informações oriundas da mesma devem ser reconhecidas e referenciadas''.



%%%%%%%%%%%%%%%%%%%%%%%%%%%%%%%%%%%%%%%%%%%%%%%%%%%%%%%%%%%%%%%%%%%%%%%%%%%
%%%%%%%%%%%%%%%%%%     The acknowledgements page         %%%%%%%%%%%%%%%%%%  
%%%%%%%%%%%%%%%%%%%%%%%%%%%%%%%%%%%%%%%%%%%%%%%%%%%%%%%%%%%%%%%%%%%%%%%%%%%
%\chapter*{Agradecimentos}
%\addcontentsline{toc}{chapter}{\numberline{}Acknowledgements}
%Thank to someone who prepared this template. Thank to someone who
%prepared this template. Thank to someone who prepared this template.
%Thank to someone who prepared this template. Thank to someone who
%prepared this template. Thank to someone who prepared this template.
%Thank to someone who prepared this template. Thank to someone who
%prepared this template. Thank to someone who prepared this
%template.Thank to someone who prepared this template. Thank to someone
%who prepared this template. Thank to someone who prepared this
%template. Thank to someone who prepared this template. Thank to
%someone who prepared this template. Thank to someone who prepared this
%template. Thank to someone who prepared this template. Thank to
%someone who prepared this template.

%%%%%%%%%%%%%%%%%%%%%%%%%%%%%%%%%%%%%%%%%%%%%%%%%%%%%%%%%%%%%%%%%%%%%%%%%%%
%%%%%%%%    tableofcontents, listoffigures and listoftables       %%%%%%%%%
%%%%%%%%        Command if you do not have  them                  %%%%%%%%%
%%%%%%%%%%%%%%%%%%%%%%%%%%%%%%%%%%%%%%%%%%%%%%%%%%%%%%%%%%%%%%%%%%%%%%%%%%%
\tableofcontents
\listoffigures
\listoftables
\clearpage


%%%%%%%%%%%%%%%%%%%%%%%%%%%%%%%%%%%%%%%%%%%%%%%%%%%%%%%%%%%%%%%%%%%%%%%%%%%
%%%%%%%%%%%%%%%%%%%%%%   END OF FRONT PAGE %%%%%%%%%%%%%%%%%%%%%%%%%%%%%%%%
%%%%%%%%%%%%%%%%%%%%%%%%%%%%%%%%%%%%%%%%%%%%%%%%%%%%%%%%%%%%%%%%%%%%%%%%%%%









