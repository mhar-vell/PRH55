%\setcounter{equation}{0}
\chapter{Introdução}

A durabilidade das estruturas em ambiente sub-aquático é de suma importância para uma produção contínua de petróleo, assim como a segurança dos funcionários e do ambiente onde as plataformas e/ou navios estão instalados. O desgaste natural ou acidental dessas instalações deve ser evitado ao máximo pois a operação de extração de petróleo é extremamente perigosa e uma falha ou imperícia pode ocasionar um desastre ambiental.

Uma plataforma de petróleo é normalmente projetada para 25 anos, tempo este que corresponde ao período de esgotamento das reservas de hidrocarbonetos, segundo \cite{1}. No entanto mesmo depois deste período algumas plataformas ainda continuam em funcionamento, consequentemente a velocidade de degradação tem um aumentam de forma exponencial devido a modificação das propriedades dos materiais utilizados nessas estruturas. Nesta situação observa-se uma extrema necessidade de um monitoramento constante das condições de uso destas instalações submersas.    
 
Atualmente, as instalações sub-aquáticas são inspecionadas por ROVs, AUVs ou mergulhadores. Estes métodos de inspeção tem um custo elevado
e sua execução utiliza muito tempo pois necessita de um navio de apoio com o AUV, ROV ou mergulhadores, pessoal qualificado e autorizado para a realização da inspeção além do tempo de mobilização, locomoção e preparo dos equipamentos e estruturas que serão necessárias para a realização do serviço
 

Um dos principais desafios na inspeção de instalações submarinas esta nos métodos aplicados atualmente que eleva os custos e o tempo para a realização de ações básicas, ações essas que poderiam ser realizadas por pessoas qualificadas da própria plataforma de petróleo, onde também ficaria o ROV aumentando a possibilidade de uma vistoria mais continua, reduzindo os custos e o tempo excluindo a utilização de um navio de apoio.
 


\section{Objetivos}
O objetivo do projeto é desenvolver um supervisório capaz de operar o OpenROV através de um computador com uma interface operacional de fácil compreensão e utilização, tendo algumas operações realizadas pelo ROV sendo executadas automaticamente, mediante a uma programação prévia e analise de informações captadas por sensores. 

\section{Objetivos específicos}
Seu desenvolvimento será realizado em dois anos de acordo com as seguintes Fases:
\textbf{\large 1ª Fase.}
 
\begin{itemize}

	\item Montagem do OpenROV.
	\item Pesquisar configuração padrão do OpenRov.
	\item Pesquisar linguagens usadas no OpenRov.
	\item Pesquisar padrões de comunicação do OpenRov.
	\item Desenvolver um algorítimo no ROS para controlar o OpenROV através de um computador.
	\item Desenvolver um algorítimo no ROS que recebe as informações adquiridas pelo sensores do OpenROV.
\end{itemize}

 \textbf{\large 2ª Fase.}
 
\begin{itemize}

	\item Desenvolver um algorítimo no ROS que controle o OpenROV de maneira semi-autônoma de acordo com as informações adquiridas.
    \item Desenvolver o ambiente de simulação no software GAZEBO.
	\item Desenvolver o modelo de simulação do OpenROV no software GAZEBO.
	\item Realizar simulações e comparar com as simulações realizadas no GAZEBO.
	\item Desenvolver o TCC.

\end{itemize}
