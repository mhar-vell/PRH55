%\setcounter{equation}{0}
\chapter{Introdução}
A integridade da infraestrutura offshore é uma questão-chave para a produção ininterrupta de petróleo e gás, bem como para a segurança dos funcionários embarcados. O envelhecimento das infraestruturas causada principalmente pela corrosão e fadiga gera um aumento da probabilidade de falha. Em casos extremos, a perda da integridade estrutural pode causar o colapso da estrutura resultando na perda completa da instalação e ocasionalmente em um desastre ecológico e humano. 
O projeto de vida típico de uma plataforma é de 20 - 25 anos, no entanto, devido à demanda atual para a produção de petróleo e gás, muitas dessas plataformas continuarão em operações além do projetado. A inspeção estrutural da região submersa destas estruturas apresenta um desafio logístico dado o meio subaquático. Logo, a importância do desenvolvimento de sistemas submarinos adequados para inspeções periódicas de integridade estrutural se faz necessário. Um agravante a este problema é o aumento do percentual da produção executada por equipamentos submersos, que resulta em uma maior demanda de inspeção periódica dos equipamentos posicionados no fundo marinho.  
Atualmente, as estruturas subaquáticas são inspecionados ou por ROVs ou mergulhadores. Ambos os métodos são dispendiosos e abrangem um tempo considerável de inspeção, principalmente por três razões: 
\begin{itemize}
	\item a necessidade de um navio de apoio,
	\item a necessidade de pessoal altamente especializado e,
	\item o tempo necessário para realizar estas inspeções essencialmente manual.
\end{itemize}

Portanto, o principal desafio no problema de inspeção estrutural submarina reside no desenvolvimento de um método eficiente de inspeção, que não necessita de um pessoal altamente especializado e que pode ser utilizada diretamente a partir das plataformas, sem a necessidade de uma embarcação de suporte.  

\section{Objetivos}
O objetivo do projeto é desenvolver um protótipo funcional de um ROV, capaz de realizar inspeção estrutural de uma plataforma offshore, onde o mesmo possa realizar algumas operações de forma autônoma e que pode ser operado diretamente de instalações fixas ou flutuantes, sem a necessidade de operadores especializados e/ou navios de apoio. Assim resultando na redução do custo de operação e manutenção, na perda de produção não planejada e nos riscos de integridade de ativos. 

\section{Objetivos específicos}
O objetivo do projeto será o desenvolvimento, em dois anos, de ROV nacional tendo algumas operações semi-autônomas, com as seguintes características:
\begin{itemize}
\item funcionamento independente de operadores;
\item propulsão realizada com motores elétricos, movidos por bateria	;
\item câmeras de espectro visível para inspeção visual da tubulação;
\item características mecânicas compatíveis com os movimentos necessários para o deslocamento subaquático;
\item sistema mecânica de baixo peso e design que permita um procedimentos simples e seguro.
\end{itemize}

\noindent Neste quesito, a busca para a completude do objetivo principal se faz necessário o alcance de alguns objetivos específicos:
\begin{itemize}
\item pesquisa e levantamento bibliográfico e do estado da arte: framework ROS, simulador GAZEBO, sistema mecânico a ser utilizado, ambiente de atuação do robô;
\item definição dos sistemas de localização, sensoriamento e motorização;
\item desenvolvimento da arquitetura de hardware e software do robô;
\item definição dos sensores e atuadores;
\item definição das ações do robô e das inspeções a serem realizadas;
\item modelagem do ambiente e do robô;
\item desenvolvimento do módulo de controle;
\item proposição de um algoritmo de controle para detecção de objetos;
\item proposição de um algoritmo de controle para transposição de objetos;
\item simulação do robô no ambiente de atuação;
\item produção de dois artigos a serem apresentado em congressos.
\end{itemize}














