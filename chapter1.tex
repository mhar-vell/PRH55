%\setcounter{equation}{0}
\chapter{Introdução}

O tempo de vida das estruturas em ambiente sub-aquático é de suma importância para uma produção contínua de petróleo. O desgaste acidental dessas instalações deve ser minimizado pois a operação de extração de petróleo é perigosa e uma falha ou imperícia pode ocasionar um desastre ambiental \cite{0}.

Uma plataforma de petróleo é normalmente projetada para 25 anos, tempo este que corresponde ao período de esgotamento das reservas de hidrocarbonetos, segundo \cite{1}. No entanto mesmo depois deste período algumas plataformas ainda continuam em funcionamento, consequentemente a taxa de degradação tem um aumento devido à modificação das propriedades dos materiais utilizados nessas estruturas. Nesta situação observa-se a necessidade de um monitoramento constante das condições de uso destas instalações submersas.    
 
Atualmente, os métodos de inspeção tem um custo elevado e sua execução utiliza muito tempo, pois dependem de um navio de apoio com o AUV, ROV, mergulhadores e pessoas qualificadas para a realização da inspeção. Outro fator oneroso é o tempo de mobilização, locomoção e preparo dos equipamentos e estruturas que serão necessárias para a realização do serviço.
 
As ações de inspeção poderiam ser realizadas por pessoas qualificadas da própria plataforma de petróleo, onde também ficaria o ROV. Desta forma aumenta-se a possibilidade de uma vistoria mais contínua, reduzindo os custos e o tempo para a execução de atividades, excluindo a utilização de um navio de apoio e o tempo perdido com mobilização e locomoção de pessoas e equipamentos.
 


\section{Objetivos}
O objetivo do projeto é desenvolver uma interface operacional de fácil compreensão e utilização, capaz de operar o OpenROV através de um computador, tendo algumas operações como o retorno a posição inicial e uma vistoria visual ao redor do objeto desejado realizadas pelo ROV sendo executadas automaticamente, mediante uma programação prévia e analise de informações captadas por sensores. 

\section{Objetivos específicos}
O desenvolvimento será realizado em dois anos de acordo com as seguintes Fases:
\textbf{ 1ª Fase.}

\begin{itemize}

	\item Criar uma pagina na internet sobre o projeto.
	\item Desenvolver um algorítimo no ROS para controlar o OpenROV através de um computador e receber as informações adquiridas dos sensores do OpenROV.
	\item Criar uma interface gráfica intuitiva para controlar o ROV.
\end{itemize}\\
\textbf{ 2ª Fase.}


\begin{itemize}

	\item Adicionar a programação realizadas no ROS os algorítimos que controlam o OpenROV de maneira semi-autônoma de acordo com os dados adquiridos por sensores.
    \item Desenvolver o ambiente aquático para a simulação do ROV no software GAZEBO.
	\item Desenvolver o modelo de simulação do OpenROV no software GAZEBO.
	\item Realizar os testes no tanque com o OpenROV e comparar com as simulações realizadas no GAZEBO.
	\item Desenvolver um artigo científico referente ao projeto.

\end{itemize}


