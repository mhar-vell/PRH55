\setcounter{equation}{0}
\chapter{Revisão bibliográfica}
EM 1953 o francês Dimitri Rebikoff desenvolveu o primeiro ROV que foi chamado de Poodle e foi utilizado em pesquisas arqueológicas subaquática \cite{6}. Desde então esta tecnologia robótica tem evoluído tanto em características construtivas, quanto em áreas de atuação podendo ser aplicados na arqueologia, na analise do ambiente marinho, na extração de petróleo e gás entre outras.

No livro "The ROV Manual: A User Guide for Remotely Operated Vehicle" o autor \cite{7} define uma classificação dos ROVs por tamanho, de acordo com a seguinte ordem:

\begin{itemize}

	\item ROVs de inspeção ou OCROV (Observation Class/Classe de Observação), geralmente são pequenos, equipados com câmeras e fontes de alimentação de corrente contínua, pesão até 100kg, e alcançam 300 metros de profundidade sendo classificados também como ROVs de baixo custo.
    \item ROVs de médio porte ou MSROV (Mid-Sized ROV/ROV médio), pesão de 100kg a 1000kg, utilizam fontes de energia com corrente alternada, e podem chegar a 3000 metros de profundidade.
	\item ROVs para a realização de trabalho ou WCROV (Work Class ROV), geralmente utilizam uma fonte de alta tensão, de 3000 volts ou mais que é convertida em energia hidráulica manipulando as ferramentas e realizando a movimentação do ROV no ambiente aquático.
	\item ROVs para operações específicas são veículos de inspeção ou de realização de trabalho que não são capazes de se locomover dentro da água sem a ajuda de um rebocador ou se locomovem caminhando sobre o solo marinho.

\end{itemize}



\section{Projetos de relevância desenvolvidos}
\begin{itemize}

	\item Projeto Flet-Fish um AUV desenvolvido pelo Instituto Brasileiro de Robótica - BIR / Senai Cimatec em parceria com o Centro de Pesquisa Alemão para Inteligência Artificial - DFKI \cite{5}.
	\item Projeto "DESENVOLVIMENTO DE UM SIMULADOR PARA VEÍCULOS SUBAQUÁTICOS COM INTERFACE 3D E ANÁLISE DE CONFIGURAÇÕES DE PROPULSORES" de \cite{6}.

\end{itemize}



 
