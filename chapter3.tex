\setcounter{equation}{0}
\chapter{Revisão bibliográfica}
De acordo com as informações obtidas com os artigos \cite{2} e \cite{3} pode-se observar vários conceitos importantes para o desenvolvimento eficiente do controle de posicionamento e movimentação de um veiculo operado remotamente em ambiente aquático observando algumas características como avanço, força, velocidade e quais fatores as influencia. 
O GAZEBO é um simulador 3D muito poderoso e tem seu código aberto como indicado \cite{4} reduzindo custos do processo de desenvolvimento das simulações virtuais sem diminuir a qualidade das simulações. 
A fundamentação teórica e a idealização deste projeto são baseados como dito anteriormente em alguns projetos desenvolvidos pelo Brazilian Institute of Robotics - BIR / Senai Cimatec com maior fundação no projeto Flat Fish que visa reduzir os custos operacionais das inspeções de estruturas subaquáticas.\cite{5}
Para este projeto tense as seguintes especificações do OpenROV além da utilização do manual de montagem e configuração fornecidos pelo fabricante através do site http://www.openrov.com/products/2-8.html:

\textbf{\large Especificações físicas:}
 
\begin{itemize}

	\item Peso 2,6 kg.
	\item Dimensões 30cm de comprimento x 20cm de largura x 15cm.
	\item Vida útil da bateria nominal usando baterias recarregáveis de lítio é de 2-3 horas (dependendo do uso).
	
\end{itemize}

\textbf{\large Especificações de performance:}
 
\begin{itemize}

	\item Profundidade máxima 100m (328 pés).
	\item O comprimento máximo de corda 300m (100m corda fornecida).
	\item Velocidade máximo 2 nós.
	\item LED brilho 200lm.
	\item Temperatura de trabalho de 10-50ºc.
	\item Software de controlado de inclinação da câmera (+/- 60 graus do centro).
	\item Pesquisar padrões de comunicação do OpenRov.
	\item Desenvolver um algorítimo no ROS para controlar o OpenRov através de um computador.
	\item Desenvolver um algorítimo no R
	
\end{itemize}


\textbf{\large Instruções de uso:}
 
\begin{itemize}

	\item Webcam HD (120 graus FOV) com áudio.
	\item Lasers Escala vermelha (paralela, 10 centímetros de separação).
	\item Proteção de corrente e tensão com feedback para garantir o funcionamento adequado do sistema.
	\item I2C bus externo com alimentação 3.3V para instrumentos externos.
	\item 6 fios auxiliares adicionais para instrumentação ou dispositivos externos definidos pelo usuário. Um canal de alimentação PWM e um canal de controle servo são pré-configurados.
	\item Código aberto.
	\item Beaglebone Black e Arduino MEGA para uma plataforma de desenvolvimento flexível e poderosa com dezenas de canais de entrada / saída e abundância de poder de computação para os recursos projetados pelo usuário.
	área de carga útil de hardware ou equipamento adicional.
	\item Área de carga útil de hardware ou equipamento adicional.
	\item Desenvolver um algorítimo no R
	
\end{itemize}

\textbf{\large Requisitos Mínimos do Sistema e equipamentos:}
 
\begin{itemize}

	\item OS / X / Windows / Linux.
	\item Últimas navegador Chrome.
	\item Um computador portátil com conector RJ-45 Ethernet e uma porta USB livre.
	\item Bateria recarregável 26650 e carregador.
	\item Um controle Gamepad, tais como:Logitech F310.
	
	
\end{itemize}




\section{Projetos de relevância desenvolvidos}

\cite{4}


